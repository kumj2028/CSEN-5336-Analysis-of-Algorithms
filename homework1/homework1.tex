%%%%%%%%%%%%%%%%%%%%%%%%%%%%%%%%%%%%%%%%%%%%%%%%%%%%%%%%%%%%%%%
%
% Welcome to writeLaTeX --- just edit your LaTeX on the left,
% and we'll compile it for you on the right. If you give
% someone the link to this page, they can edit at the same
% time. See the help menu above for more info. Enjoy!
%
%%%%%%%%%%%%%%%%%%%%%%%%%%%%%%%%%%%%%%%%%%%%%%%%%%%%%%%%%%%%%%%

% --------------------------------------------------------------
% This is all preamble stuff that you don't have to worry about.
% Head down to where it says "Start here"
% --------------------------------------------------------------
 
\documentclass[12pt]{article}
 
\usepackage[margin=1in]{geometry}
\usepackage{amsmath,amsthm,amssymb}

\usepackage{listings}
\usepackage{xcolor}

%New colors defined below
\definecolor{codegreen}{rgb}{0,0.6,0}
\definecolor{codegray}{rgb}{0.5,0.5,0.5}
\definecolor{codepurple}{rgb}{0.58,0,0.82}
\definecolor{backcolour}{rgb}{0.95,0.95,0.92}

%Code listing style named "mystyle"
\lstdefinestyle{mystyle}{
  backgroundcolor=\color{backcolour}, commentstyle=\color{codegreen},
  keywordstyle=\color{magenta},
  numberstyle=\tiny\color{codegray},
  stringstyle=\color{codepurple},
  basicstyle=\ttfamily\footnotesize,
  breakatwhitespace=false,         
  breaklines=true,                 
  captionpos=b,                    
  keepspaces=true,                 
  numbers=left,                    
  numbersep=5pt,                  
  showspaces=false,                
  showstringspaces=false,
  showtabs=false,                  
  tabsize=2
}

%"mystyle" code listing set
\lstset{style=mystyle}

 
\newcommand{\N}{\mathbb{N}}
\newcommand{\Z}{\mathbb{Z}}
 
\newenvironment{theorem}[2][Theorem]{\begin{trivlist}
\item[\hskip \labelsep {\bfseries #1}\hskip \labelsep {\bfseries #2.}]}{\end{trivlist}}
\newenvironment{lemma}[2][Lemma]{\begin{trivlist}
\item[\hskip \labelsep {\bfseries #1}\hskip \labelsep {\bfseries #2.}]}{\end{trivlist}}
\newenvironment{exercise}[2][Exercise]{\begin{trivlist}
\item[\hskip \labelsep {\bfseries #1}\hskip \labelsep {\bfseries #2.}]}{\end{trivlist}}
\newenvironment{problem}[2][Problem]{\begin{trivlist}
\item[\hskip \labelsep {\bfseries #1}\hskip \labelsep {\bfseries #2.}]}{\end{trivlist}}
\newenvironment{question}[2][Question]{\begin{trivlist}
\item[\hskip \labelsep {\bfseries #1}\hskip \labelsep {\bfseries #2.}]}{\end{trivlist}}
\newenvironment{corollary}[2][Corollary]{\begin{trivlist}
\item[\hskip \labelsep {\bfseries #1}\hskip \labelsep {\bfseries #2.}]}{\end{trivlist}}

\newenvironment{solution}{\begin{proof}[Solution]}{\end{proof}}
 
\begin{document}
 
% --------------------------------------------------------------
%                         Start here
% --------------------------------------------------------------
 
\title{Homework 1}%replace X with the appropriate number
\author{Mengxiang Jiang\\ %replace with your name
CSEN 5336 Analysis of Algorithms} %if necessary, replace with your course title
 
\maketitle
 
\begin{problem}{1} %You can use theorem, exercise, problem, or question here.  Modify x.yz to be whatever number you are proving
    Some functions are given below. Sort them in ascending order of asymptotic growth (big-O). (lg is log function with base 2)\\
    1. $5\lg{n}$\\
    2. $6n\lg{n}$\\
    3. $n^{n/8}$\\
    4. $7\lg{\lg{n}}$\\
    5. $n^{0.6}$\\
    6. $2n^{\lg{n}}$\\
    7. $\lg^{12}{n}$ [or $(\lg{n})^{12}$]\\
    8. $(n/2)^n$ \\
    9. $n^{0.5}\lg{n}$\\
    10. $3n$\\
    \\
    $7\lg{\lg{n}} < 5\lg{n} < \lg^{12}{n} < n^{0.5}\lg{n} < n^{0.6} < 3n < 6n\lg{n} < 2n^{\lg{n}} < n^{n/8} < (n/2)^n$
\end{problem}

\begin{problem}{2}
    Solve the recurrence relations using the master method\\
    a) $T(n) = 2T(n/2) + 3n$\\
    b) $T(n) = 2T(n/4) + 4n^{0.3}$\\
    c) $T(n) = 2T(n/2) + 2n^2$\\
    d) $T(n) = 3T(n/3) + 3n\lg{n}$\\
    e) $T(n) = 2T(n/2) + \Theta(n)$\\\\
    a) $a = 2$, $b = 2$, $f(n) = 3n$, $\log_b{a} = \log_2{2} = 1$, $n^{\log_b{a}} = n^1 = \Theta(n) = \Theta(f(n))$\\
    since $f(n)$ is the same size as $n^{\log_b{a}}$, case 2 applies and $T(n) = \Theta(n^{\log_b{a}}\log{n}) = \Theta(n\log{n})$\\\\
    b) $a = 2$, $b = 4$, $f(n) = 4n^{0.3}$, $\log_b{a} = \log_4{2} = 0.5$\\ 
    $n^{\log_b{a}} = n^{0.5} = \Theta(n^{0.5}) > \Theta(f(n)) = \Theta(n^{0.3})$\\
    since $f(n)$ is polynomially smaller than $n^{\log_b{a}}$, case 1 applies and $T(n) = \Theta(n^{\log_b{a}}) = \Theta(n^{0.5})$\\\\
    c) $a = 2$, $b = 2$, $f(n) = 2n^2$, $\log_b{a} = \log_2{2} = 1$, $n^{\log_b{a}} = n^1 = \Theta(n) < \Theta(f(n)) = \Theta(n^2)$\\
    since $f(n)$ is polynomially larger than $n^{\log_b{a}}$, case 3 applies and $T(n) = \Theta(n^2)$\\\\
    d) $a = 3$, $b = 3$, $f(n) = 3n\lg{n}$, $\log_b{a} = \log_3{3} = 1$\\
    $n^{\log_b{a}} = n^{1} = \Theta(n) < \Theta(f(n)) = \Theta(n\log{n})$\\
    since $f(n)$ is larger but not polynomially larger than $n^{\log_b{a}}$, $T(n)$ is not solvable by the master theorem\\\\
    e) $a = 2$, $b = 2$, $f(n) = \Theta(n)$, $\log_b{a} = \log_2{2} = 1$, $n^{\log_b{a}} = n^1 = \Theta(n) = \Theta(f(n))$\\
    since $f(n)$ is the same size as $n^{\log_b{a}}$, case 2 applies and $T(n) = \Theta(n^{\log_b{a}}\log{n}) = \Theta(n\log{n})$\\\\
\end{problem}
 
\begin{problem}{3}
    Calculate the running time of the algorithms using big-O notation:\\
    a) \begin{verbatim}
    for (i = 1; i*i<n; i++)
        printf(“%d\n”, i)
    \end{verbatim}
    b) \begin{verbatim}
    for (i = n; i > 1; i = ceil(i/8))
        printf(“%f\n”, i);
    \end{verbatim}
    a) Since $n$ is compared with $i*i = i^2$, and as $i$ grows linearly, $i^2$ grows quadratically, so the time it will take for $i^2$ to reach $n$ is $\sqrt{n}$, so $T(n) = O(\sqrt{n}) = O(n^{0.5})$\\\\
    b) Since $i$ is divided by 8 each iteration, starting with $i = n$, the amount of time for $i$ to reach 1 or lower is $\log_8{n}$, so $T(n) = O(\log(n))$
\end{problem}

% --------------------------------------------------------------
%     You don't have to mess with anything below this line.
% --------------------------------------------------------------
 
\end{document}